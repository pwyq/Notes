\documentclass[lang=en,mode=geye,device=normal,color=blue,14pt]{elegantnote}
\usepackage{amsmath}
\usepackage{amssymb}
\usepackage{bbm}
\usepackage{tcolorbox}
\usepackage{graphicx}
\usepackage{booktabs}
\usepackage{subfigure}
%\usepackage{subfig}
\usepackage{subfiles}
\usepackage{bm}
\usepackage{pythonhighlight}

\usepackage{xcolor}
\definecolor{shadecolor}{RGB}{150,150,150}
\newcommand{\mybox}[1]{\par\noindent\colorbox{shadecolor}
{\parbox{\dimexpr\textwidth-2\fboxsep\relax}{#1}}}

\DeclareMathOperator*{\E}{\mathbb{E}}
\DeclareMathOperator*{\A}{\mathcal{A}}
%\DeclareMathOperator*{\S}{\mathcal{S}}
\DeclareMathOperator*{\1}{\mathbbm{1}}
\DeclareMathOperator*{\R}{\mathbbm{R}}
\DeclareMathOperator*{\argmax}{arg\,max}
\DeclareMathOperator*{\argmin}{arg\,min}

\title{Notes of Reinforcement Learning: An Introduction}

\author{Yanqing Wu}
%\institute{Viwistar Robotics}

% \version{0.1.0}
\date{\today}

\begin{document}
\maketitle

\setlength{\parindent}{0pt}
\tableofcontents
%%%%%%%%%%%%%%%%%%%%%%%

\subfile{chapter_1.tex}

%%%%%%%%%%%%%%%%%%%%%%%

\subfile{chapter_2.tex}

%%%%%%%%%%%%%%%%%%%%%%%

\subfile{chapter_3.tex}

%%%%%%%%%%%%%%%%%%%%%%%

\subfile{chapter_4.tex}

%%%%%%%%%%%%%%%%%%%%%%%

\subfile{chapter_5.tex}

%%%%%%%%%%%%%%%%%%%%%%%

\subfile{chapter_6.tex}

% TODO: rest of chapter 6
% TODO: chap7
\subfile{chapter_7.tex}

%%%%%%%%%%%%%%%%%%%%%%%

\subfile{chapter_8.tex}

%%%%%%%%%%%%%%%%%%%%%%%
% PART II
%%%%%%%%%%%%%%%%%%%%%%%

\subfile{chapter_9.tex}

%%%%%%%%%%%%%%%%%%%%%%%

\subfile{chapter_10.tex}

\newpage
\section{Off-policy Methods with Approximation}

\newpage
\section{Eligibility Traces}
%%%%%%%%%%%%%%%%%%%%%%%

\subfile{chapter_13.tex}

%%%%%%%%%%%%%%%%%%%%%%%%%%%%%%%%

\subfile{chapter_extra_notes.tex}

%%%%%%%%%%%%%%%%%%%%%%%%%%%%%%%%

\bibliography{references.bib}

\end{document}